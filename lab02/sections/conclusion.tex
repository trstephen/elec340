\newpage
\section{Conclusion}\label{sec:conclusion}
The simulation was an effective tool to see how waves propagate in free space vs. in a dielectric.
The results of a wave propagating in free space with an animation region larger than the container demonstrate that the wave is not confined by the container (Fig. \ref{fig:non-ideal}) whereas the wave that is propagating in a dielectric is much more confined to the container (Fig. \ref{fig:Task3-3d-animation}).
Widening electric boundaries and spacing them close together was able to approximate the plane wave behavior (Fig. \ref{fig:narrow}) 

In free space the simulation allowed us to calculate the phase velocity of the wave from the change in distance over the change in time.
As expected the wave propagated at the speed of light. 
In the dielectric, we were able to calculate alpha from the ln of the ratio of the electric field amplitudes at two points divided by the distance between the  points, and $\beta$ from $2\pi$ divided by the wavelength. 

Both numbers matched the theoretical calculations.
Since the experimentally calculated $\alpha$ and $\beta$ matched the theoretical $\alpha$ and $\beta$,  the Helmholtz wave equation is experimentally verified. 
