\section{Introduction}\label{sec:intro}
Faraday's and Ampere's Laws can be applied \cite[pp. 15-16]{lab-manual} to a medium with constant permittivity and permeability to yeild the Helmholtz equation:
\begin{equation}
	\del^2 \phasor{E} = - \omega^2 \mu \epsilon \phasor{E} = \gamma^2 \phasor{E}.
\end{equation}
This relation implies that the Electric field and the magnetic field described by Faraday's and Ampere's Law form an electromagnetic wave the varies in time.
The amplitude of the waves relate by the intrinsic impedance of the material.
The intrinsic impedance is a material property that is calculated using the permittivity and permeability of the material. 

The amplitude of the waves as well as their phase difference describe their polarization.
In the most general case, each wave has a different amplitude and the phase shift is any angle between 0 and $\pi$.
This corresponds to elliptical polarization.
For specific amplitudes and phase angles correspond to special types of polarization: a phase angle of 0 or $\pi$ corresponds to linear polarization and a phase angle of $\pm 2\pi$ with equal amplitudes corresponds to circular polarization. 
